% % Title of the paper:
\newcommand{\hemaClassTitle}{\hemaClass{}: Online one-by-one normalization and classification of hematological malignacies}

% Load packages
\usepackage{fullpage} % Larger margins
\usepackage{amssymb,amsmath}
\usepackage{authblk}  % For author affiliations
\usepackage[hypertexnames=false]{hyperref} % For urls and hyperlinks
\usepackage{graphicx}
\usepackage[numbers,sort]{natbib}
\usepackage{cite} % Make references as [1-4], not [1,2,3,4]

% To do notes
\usepackage[
%  disable, %turn off todonotes
  colorinlistoftodos, %enable a coloured square in the list of todos
  textwidth=2cm, %set the width of the todonotes
  textsize=scriptsize, %size of the text in the todonotes
  ]{todonotes}

% Macros
\newcommand{\hemaClass}{\href{http://hemaClass.org}{\texttt{hemaClass.org}}}
\newcommand{\R}{\textsf{R}}
\newcommand{\pkg}[1]{\textbf{#1}}

\DeclareMathOperator*{\median}{median}
\DeclareMathOperator*{\std}{std}

% Hypenation
\hyphenation{Chemo-resistance}


% \begin{document}

\phantomsection
\addcontentsline{toc}{section}{Supplementary Material}
\begin{center}
{\huge SUPPLEMENTARY MATERIAL}\bigskip \\
{\bf \hemaClassTitle{}}
\end{center}

\section{Graham's formula}
This section derives the Graham's formula which, in our context, yield the posterior probability of being resistant to the combination of two drugs, given resistance to the individual drugs.
For simplicity, the formula is derived for two drugs.
The formula straightforwardly generalizes to three or more drugs.

Let $C$, $H$, and $B$ be Bernoulli distributed random variables with probability parameter $1/2$, where
$C = 1$ indicates resistance to drug Cyclophosphamide $C$,
$H = 1$ indicates resistance to drug Doxorubicin $H$, and
$B = 1$ indicates resistance to the combination of drug $H$ and $C$.
Conversely, $C = 0$ indicates sensitivity towards the drug $C$.
Under an assumption of conditional drug independence
\begin{align*}
  P(C=1, H=1| B=1) &= P(C=1 | B=1) P(H=1 | B=1), \text{ and } \\
  P(C=1, H=1| B=0) &= P(C=1 | B=0) P(H=1 | B=0)
\end{align*}
we have that
\begin{align*}
  &P(B=1 | H=1, C=1)
  \\&\qquad
   = \frac{P(C=1, H=1, B=1)}
          {P(C=1, H=1)}
  \\&\qquad
   = \frac{P(C=1, H=1 | B=1) P(B=1)}
          {P(C=1, H=1, B=1) + P(C=1, H=1, B=0)}
  \\&\qquad
   = \frac{P(C=1 | B=1) P(H=1 | B=1) P(B=1)}
          {P(C=1, H=1 | B=1) P(B=1) + P(H=1, C=1| B=0) P(B=0)},
\end{align*}
by the definition of conditional probabilities, the law of total probability, and the assumptions.
From the distributional assumption on $B$, $P(B=0) = P(B=1) = 1/2$, and the above then simplifies to:
\begin{equation*}
  P(B=1 | H=1, C=1)
   = \frac{P(C=1 | B=1) P(H=1 | B=1)}
          {P(C=1, H=1 | B=1) + P(H=1, C=1 | B=0)}.
\end{equation*}
For notational convenience, we abbreviate
$p_C = P(C=1 | B=1)$,
$p_H = P(H=1 | B=1)$,
$p_{CH} = P(B=1 | H=1, C=1)$.
The distributional assumptions then imply:
\begin{align*}
  p_{CH}
  &= \frac{p_C p_H}
          {p_C p_H + P(C=1 | B=0) P(H=1 | B=0)}
  \\
  &= \frac{p_C p_H}
          {p_C p_H + P(B=0 | C=1) P(B=0 | H=1)}
  \\
  &= \frac{p_C p_H}
          {p_C p_H + \bigl(1 - P(B=1 | C=1)\bigr)\bigl(1 - P(B=1 | H=1)\bigr)}
  \\
  &= \frac{p_C p_H}
          {p_C p_H + (1 - p_C)(1 - p_H)},
\end{align*}
which is the two-drug equivalent to the used formula.

% \end{document}
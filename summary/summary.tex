%%%%%%%%%%%%%%%%%%%%%%%%%%%%
% Title of the paper:
\newcommand{\hemaClassTitle}{\hemaClass{}: Online one-by-one normalization and classification of hematological malignacies}

% Load packages
\usepackage{fullpage} % Larger margins
\usepackage{amssymb,amsmath}
\usepackage{authblk}  % For author affiliations
\usepackage[hypertexnames=false]{hyperref} % For urls and hyperlinks
\usepackage{graphicx}
\usepackage[numbers,sort]{natbib}
\usepackage{cite} % Make references as [1-4], not [1,2,3,4]

% To do notes
\usepackage[
%  disable, %turn off todonotes
  colorinlistoftodos, %enable a coloured square in the list of todos
  textwidth=2cm, %set the width of the todonotes
  textsize=scriptsize, %size of the text in the todonotes
  ]{todonotes}

% Macros
\newcommand{\hemaClass}{\href{http://hemaClass.org}{\texttt{hemaClass.org}}}
\newcommand{\R}{\textsf{R}}
\newcommand{\pkg}[1]{\textbf{#1}}

\DeclareMathOperator*{\median}{median}
\DeclareMathOperator*{\std}{std}

% Hypenation
\hyphenation{Chemo-resistance}


%%%%%%%%%%%%%%%%%%%%%%%%%%%%

\title{\hemaClassTitle{}}

\author[1]{\small Steffen Falgreen\thanks{\texttt{sfl@rn.dk}}}
\author[12]{Anders Ellern Bilgrau\thanks{\texttt{anders.ellern.bilgrau@gmail.com}}}
\author[12]{Lasse Hjort Jakobsen\thanks{\texttt{lasse.j@rn.dk}}}
\author[12]{Jonas Have\thanks{\texttt{jonas.have@rn.dk}}}
\author[12]{Kasper Lindblad Nielsen\thanks{\texttt{k.lindblad@rn.dk}}}
\author[1]{Tarec Christoffer El-Galaly\thanks{\texttt{tarec.galaly@gmail.com}}}
\author[1]{Julie St{\o}ve  B{\o}dker\thanks{\texttt{j.boedker@rn.dk}}}
\author[1]{Alexander Schmitz\thanks{\texttt{alex.schmitz@rn.dk}}}
\author[3]{Ken H. Young\thanks{\texttt{khyoung@mdanderson.org}}}
\author[12]{Hans Erik Johnsen\thanks{\texttt{haej@rn.dk}}}
\author[12]{Karen Dybk{\ae}r\thanks{\texttt{k.dybkaer@rn.dk}}}
\author[12]{Martin B{\o}gsted\thanks{\texttt{martin.boegsted@rn.dk}}}

\affil[1]{Department of Haematology, Aalborg University Hospital}
\affil[2]{Department of Clinical Medicine, Aalborg University}
\affil[3]{Department of Hematopathology, MD Anderson Cancer Center}
\date{\small \today}


\begin{document}

\maketitle
\textbf{Keywords:~}
Diffuse Large B-Cell Lymphoma; Classification; Chemosensitivity; Gene expression profiling \\

Dozens of genomics based cancer classification systems have been introduced with prognostic, diagnostic, and predictive capabilities.
However, they often employ complex algorithms and are only applicable on whole cohorts of patients, making them difficult to apply in a personalized clinical setting. This prompted us to create \hemaClass{}, available at \url{http://hemaclass.org}, which is an online web application providing an easy interface to one-by-one microarray based risk classifications of diffuse large B-cell lymphoma (DLBCL) into cell-of-origin and chemotherapeutic sensitivity classes.

Users upload patient samples which are normalized and classified depending on settings chosen by the users. The server accepts samples on the Affymetrix GeneChip HG-U133 Plus 2.0 array. Three  methods of RMA normalization is available, cohort based, reference based and one-by-one normalization. The classifications available on \hemaClass{} are:
(1) ABC/GCB \cite{Alizadeh2000} which is a classification of DLBCL cases to either activated B-cell phenotype (ABC) or germinal center B-cell phenotype (GCB),
(2) BAGS \cite{DybkaerBoegsted2015} which is a refinement of the ABC/GCB classification of DLBCL capable of classifying DLBCL samples into $5$ different B-cell subtypes,
(3) REGS \cite{Falgreen2015} which is a classification based resistance gene signatures (REGS) for the most prominent drugs used in the treatment of DLBCL patients.
The results are presented on the webserver, and is available for download. An optional text summary of the results and a plot of the predicted progression free survival and overall survival is provided for each patient sample.

The server has been running since August 2014, and has been validated satisfactory on four different DLBCL cohorts with a total of 882 patients, e.g. the ABC/GCB classification shows an approximate 90\% intermethod agreement on all four validation sets. None of the validation sets have been used in the training of the classifiers. The server has been in routine use in our lab for classification since it was initiated, e.g. the paper by Marquez et al. \cite{}.

\newpage
\bibliographystyle{plain}
\bibliography{references}
\end{document}

